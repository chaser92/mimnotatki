% !TEX TS-program = pdflatex
% !TEX encoding = UTF-8 Unicode

% Example of the Memoir class, an alternative to the default LaTeX classes such as article and book, with many added features built into the class itself.

%\documentclass[12pt,a4paper]{memoir} % for a long document
\documentclass[12pt,a4paper,article]{memoir} % for a short document
\usepackage{polski}
\usepackage[utf8]{inputenc} % set input encoding to utf8

% Don't forget to read the Memoir manual: memman.pdf

%%% Examples of Memoir customization
%%% enable, disable or adjust these as desired

%%% PAGE DIMENSIONS
% Set up the paper to be as close as possible to both A4 & letter:
\settrimmedsize{11in}{210mm}{*} % letter = 11in tall; a4 = 210mm wide
\setlength{\trimtop}{0pt}
\setlength{\trimedge}{\stockwidth}
\addtolength{\trimedge}{-\paperwidth}
\settypeblocksize{*}{\lxvchars}{1.618} % we want to the text block to have golden proportionals
\setulmargins{50pt}{*}{*} % 50pt upper margins
\setlrmargins{*}{*}{1.618} % golden ratio again for left/right margins
\setheaderspaces{*}{*}{1.618}
\checkandfixthelayout 
% This is from memman.pdf

%%% \maketitle CUSTOMISATION
% For more than trivial changes, you may as well do it yourself in a titlepage environment
\pretitle{\begin{center}\sffamily\huge\MakeUppercase}
\posttitle{\par\end{center}\vskip 0.5em}

%%% ToC (table of contents) APPEARANCE
\maxtocdepth{subsection} % include subsections
\renewcommand{\cftchapterpagefont}{}
\renewcommand{\cftchapterfont}{}     % no bold!

%%% HEADERS & FOOTERS
\pagestyle{ruled} % try also: empty , plain , headings , ruled , Ruled , companion

%%% CHAPTERS
\chapterstyle{hangnum} % try also: default , section , hangnum , companion , article, demo

\renewcommand{\chaptitlefont}{\Huge\sffamily\raggedright} % set sans serif chapter title font
\renewcommand{\chapnumfont}{\Huge\sffamily\raggedright} % set sans serif chapter number font

%%% SECTIONS
\hangsecnum % hang the section numbers into the margin to match \chapterstyle{hangnum}
\maxsecnumdepth{subsection} % number subsections

\setsecheadstyle{\Large\sffamily\raggedright} % set sans serif section font
\setsubsecheadstyle{\large\sffamily\raggedright} % set sans serif subsection font

%% END Memoir customization

\title{Teoria liczb}
\author{}
\date{} % Delete this line to display the current date

%%% BEGIN DOCUMENT
\begin{document}

\maketitle
\tableofcontents* % the asterisk means that the contents itself isn't put into the ToC

\chapter{Dzielenie z resztą}
\[ \delta > 0 => \forall a \exists ! q,r: a = b \dot q + r, 0 \leq r < b \]
Rozważmy zbiór:
D. $ A = {a - b \dot t, t \in Z } [zbior_oraz] N  \neq [zbiorpusty], {min} A = r $ 
Def. Wielokrotnością liczby $ b \neq 0 $ i $ a \neq b \dot c $, to a jest wielokrotnością b.
Jeśli przy tym $ b > 0 $ to b jest dzielnikiem a. (ozn $ b | a $) \\
\section{Liczby pierwsze}
Jeśli jedynym dzielnikiem $a >1$ są $1$ i $a$, to ta liczba nazywa się pierwsza. \\
W szczególności 1 i liczby ujemne nie są pierwsze.
\section{Największy wspólny dzielnik}
$a^2 + b^2 \neq 0$, ale jest też definicja odwołująca się do relacji podzielności, bez zakładania, że istnieje relacja porządku.
Jeśli $s$ jest w.d. liczb $a$ i $b (d | a,b)$ i $(d | a,b => d|s)$, to $s = NWD(a,b)$. \\
\textbf{Fakt:} NWD(a,b) istnieje i jest jednoznacznie wyznaczony.
\subsection{Ideał}
\textbf{Ideał} to niepusty zbiór X taki, że $ X [zawarty-rowny] Z $ zamknięty na +, - \\
\textbf{Podfakt} Każdy ideał w $Z$ jest generowany przez jeden element $a$, to znaczy jest \textbf{ideałem głównym}, czyli jest postaci $a \cdot x: x \in Z$
\subsection{Dowód podfaktu ideału głównego dla Z}
${0}$ to dobry ideał choć nieciekawy. Ale jeśli $X \neq {0}$, niech $ b = min X [andzbiorowy] N_+ $. No to teraz $a$ jest generatorem. Dlaczego? Przypuśćmy, że czegoś nie generuje, tzn. są elementy które nie są postaci $ ax $, to dzielimy $y$ mod $a$, ta reszta z dzielenia jest elementem ideału, a jest mniejsza niż $a$. Co jest sprzeczne z założeniami xD
\subsection{Algorytm Euklidesa i NWD - ciąg dalszy ideałów}
Niech $X = {ax + by, x,y \in Z}$ - ideał. Jeśli weźmiemy generator tego ideału, to on będzie dobry. $X$ ma generator $s$, który jest tej postaci, tj. $s=ax_0 + by_0 $. Pokażemy, że $s$ ma te 2 własności. $s|a,b$ bo $a,b \in X$. $d|a,b => d|ax_0+bx_0$.
NWD jest kombinacją liniową tych liczb, których jest wspólnym dzielnikiem. Znany jest wzór rekurencyjny na NWD (algorytm Euklidesa). 
\[NWD(a,b) = \cases{a b=0} {NWD(b,a mod b) b>0} TODO nie dziala cases \]
\subsection{Rozszerzony algorytm Euklidesa}
Oprócz NWD oblicza dodatkowo $x, y$, t. że $NWD (a,b) = ax + by$. \\
$b = 0, => (x,y) = (1,0) $ (a razy 1 plus b razy 0 równa się NWD). \\
Jeśli $ b > 0 $ to z poprzedniego wywołania rekurencyjnego znamy $x', y'$ takie, że $NWD(b, a mod b) = bx' + (a mod b)y'$. A $ a mod b $ to nic innego jak $a - b \cdot TODO podloga{\frac{a}{b}}$. Zatem y' = moje nowe x, a $a - b \cdot TODO podloga{\frac{a}{b}}$ - moje nowe y. \\
Przykład: TODO
tabelka{{a}{b}{x}{y}}
{7}{5}{-2}{3}
{5}{2}{1}{-2}
{2}{1}{0}{1}
{1}{0}{1}{0}
Umiesz już policzyć nie tylko NWD, ale też te współczynniki x, y. Nieźle! Tak trzymać! \\
Def. $ a \perp b \Leftrightarrow NWD(a,b) = 1 $ \\ a,  względnie pierwsze \\
\textbf{Fakt} $a|bc, a \perp b \Rightarrow a | c. $
\textbf{Dowód} $ a \perp b \rightarrow \exists x,y : 1 = ax+by \rightarrow c = acx + bcy $ - prawa strona jest podzielna przez a, więce lewa też. 
\subsection{Podstawowe twierdzenie arytmetyki}
Każda liczba $a>0$ ma rozkład $a=\prod_{i=1}^{m} b_i$, gdzie$ b_i $- pierwsza jednoznaczna z dokładnością do kolejności czynników. \\
\textbf{Dowód} Istnienie oczywiste - indukcja. \\
\textbf{Jednoznaczność}:
indukcja: $a=1 \rightarrow $ok. \\
Niech $ a=b_1 \cdot ... \cdot b_m = c_1 \cdot ... \cdot c_k $. Wystarczy pokazać, że $b_1 = c_1$ i skorzystać z jednoznaczności rozkładu liczby $\frac{a}{b_1}$. Przypuśćmy, że tak nie jest i załóżmy, że $b_1 < c_1 \rightarrow b_1 \perp c_1 \rightarrow \exists x, y: 1 = b_1 x + c_1 y $.
 Pomnożymy to stronami przez $c_2 \cdot ... \cdot c_k $. $b_1$ dzieli prawą stronę, więc i lewą.
$ b_1 | c_2 \cdot ... \cdot c_k \rightarrow^{(*)} b_1 | c_i$ dla pewnego $i \in {{} 2 ... k{}} $ \\
\textbf{Przykład}:  \[ a = {p_1}^{a_1} \cdot ... \cdot {p_k}^{a_k} \] 
\[ a = {p_1}^{b_1} \cdot ... \cdot {p_k}^{b_k} \] 
\[ (\alpha_i, \beta_i \geq 0 \]
\[ b | a \leftrightarrow \beta_i \leq \alpha_i \forall_i \]
\[ a \perp b \leftrightarrow min (\alpha_i, \beta_i) = 0 \forall i \]
\[ NWD (a,b) = \prod_{i} min(\alpha_i, \beta_i) \]
\subsection{Największa wspólna wielokrotność}
$ NWW (a,b) = s,$ t, że $a,b | s$ i $(a,b|d \rightarrow s | d)$.
\textbf{Definicja} $NWW (a,b) = \frac{a \cdot b}{NWD(a,b)} = \prod_i {p_i}^{max(\alpha_i, \beta_i)} $ \\
\section{Kongruencja - przystawanie} 
Określmy $n > 0$ i nazwijmy ją \em modułem kongruencji \em. \\ 
Wtedy $a === b (mod n) \leftrightarrow n | a - b$. Ten symbol czytamy: $a$ jest \em kongruentne \em z $b$. Dla dowolnie przyjętego $n$ kongruentność jest relacją równoważności. Klasy abstrakcji określa zbiór dzielenia przez $n$ - $x(n+r)$ dla $r \in {{}0,...,n-1{}} $. \\
\textbf {Własności relacji kongruencji}. Kongruencje przy tym samym module można dodawać, odejmować i mnożyć stronami. Czyli $ a \equiv b (mod n), c \equiv d (mod n) \rightarrow a_i a + - * c \equiv b + - * d (mod n). $ \\
\textbf {Przykład} $ ac - bd = ac-bc + bc-bd = (a-b)\cdot c + b \cdot (c-d)$ (a gdzie tu ta kongruencja cała ?) \\
Co z dzieleniem stronami? Otóż o ile $ d \perp b, ad \equiv bd $ (mod $n$) $ \rightarrow a \equiv b $ (mod $n$) \\
$n | d (a-b) \rightarrow n | a - b$ \\
Bez dodatkowych założeń, jeśli $ad \perp bd$ mod($nd$) $\leftrightarrow a \equiv b$ (mod $n$). \\
\textbf {Jak zamienić układ 2 kongruencji z względnie pierwszymi modułami na jedną?} Jeśli $m \perp n, a \equiv (b mod (m \cdot n) $, to $ a \equiv b (mod n), a \equiv b (mod m) $. \\
\textbf{Definicja} $ m \cdot n | x \leftrightarrow n | x ORAZ m | x $  
\subsection{Chińskie twierdzenie o resztach}
$n_1, ... n_k$ - parami względnie pierwsze \\
ozn. $n = n_1 \cdot ... \cdot n_k $ \\
\[ \forall a_1, ..., a_k \exists ! 0 \leq a < n : a \equiv a_i TODO (mod n_i)\] dla $i=1 ... k$ \\
\textbf {Przykład} $ n_i = 7, 11, 13; a_i = 5, 3, 1 \rightarrow a = 817 $. \\
\textbf{Dowód niekonstruktywny} $ {{}0 ... n-1 {}} \rightarrow <a mod n_1, ..., a mod n_k> $. \\
Wystarczy pokazać, że jeśli $ a \not\equiv b (mod n) $ to $a$ i $b$ mają różne wektory. \\
Z (**) wynika, że $ a \equiv (mod n_i) i \in {{}1...k{}} \leftrightarrow (mod n_1 \cdot ... n_k) \\
 \end{document}


