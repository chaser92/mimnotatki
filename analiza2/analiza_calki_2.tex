% !TEX TS-program = pdflatex
% !TEX encoding = UTF-8 Unicode

% Example of the Memoir class, an alternative to the default LaTeX classes such as article and book, with many added features built into the class itself.

%\documentclass[12pt,a4paper]{memoir} % for a long document
\documentclass[12pt,a4paper,article]{memoir} % for a short document
\usepackage{polski}
\usepackage[utf8]{inputenc} % set input encoding to utf8

% Don't forget to read the Memoir manual: memman.pdf

%%% Examples of Memoir customization
%%% enable, disable or adjust these as desired

%%% PAGE DIMENSIONS
% Set up the paper to be as close as possible to both A4 & letter:
\settrimmedsize{11in}{210mm}{*} % letter = 11in tall; a4 = 210mm wide
\setlength{\trimtop}{0pt}
\setlength{\trimedge}{\stockwidth}
\addtolength{\trimedge}{-\paperwidth}
\settypeblocksize{*}{\lxvchars}{1.618} % we want to the text block to have golden proportionals
\setulmargins{50pt}{*}{*} % 50pt upper margins
\setlrmargins{*}{*}{1.618} % golden ratio again for left/right margins
\setheaderspaces{*}{*}{1.618}
\checkandfixthelayout 
% This is from memman.pdf

%%% \maketitle CUSTOMISATION
% For more than trivial changes, you may as well do it yourself in a titlepage environment
\pretitle{\begin{center}\sffamily\huge\MakeUppercase}
\posttitle{\par\end{center}\vskip 0.5em}

%%% ToC (table of contents) APPEARANCE
\maxtocdepth{subsection} % include subsections
\renewcommand{\cftchapterpagefont}{}
\renewcommand{\cftchapterfont}{}     % no bold!

%%% HEADERS & FOOTERS
\pagestyle{ruled} % try also: empty , plain , headings , ruled , Ruled , companion

%%% CHAPTERS
\chapterstyle{hangnum} % try also: default , section , hangnum , companion , article, demo

\renewcommand{\chaptitlefont}{\Huge\sffamily\raggedright} % set sans serif chapter title font
\renewcommand{\chapnumfont}{\Huge\sffamily\raggedright} % set sans serif chapter number font

%%% SECTIONS
\hangsecnum % hang the section numbers into the margin to match \chapterstyle{hangnum}
\maxsecnumdepth{subsection} % number subsections

\setsecheadstyle{\Large\sffamily\raggedright} % set sans serif section font
\setsubsecheadstyle{\large\sffamily\raggedright} % set sans serif subsection font

%% END Memoir customization

\title{Analiza II}
\author{Mariusz Kierski}
% \date{} 

%%% BEGIN DOCUMENT
\begin{document}

\maketitle
\tableofcontents* % the asterisk means that the contents itself isn't put into the ToC

\chapter{Całki Riemanna i Lebesgue'a}
\section{Tw. o wartości średniej}
Jeżeli $b>a$, $f:[a;b]->R$ - ciagła, to istn. $c \in (a;b)$ t. że:
\[ \int_{[a;b]}f(x)dx = (b-a)f(c) \]
\section{Całki niewłaściwe}
Dla całki Riemanna $f : [a;b] -> R$ \\
co gdy $f:[a;b)->R$? \\
Dwa przypadki: $ b \in R $ oraz $ b = +\infty $.
Założenia wstepne dot. całki niewłaściwej:  
(*) $ \forall_{r \in [a;b)} f |_{[a;r]} $ - całkowalna w sensie Riemanna. \\
 Definicja \\ 
(całka niewłaściwa "prawostronnie") - tj. "brakuje" końca przedziału.
Jeżeli istnieje granica $ lim_{r -> b^{-}} \int_{[a;r]} f(x)dx = G $ to nazywamy ją całką niewłaściwą z $f$ i oznaczamy symbolem $\int_{a}^{b} f(x)dx $
Całka niewłaściwa jest zbieżna, gdy powyższa granica G istnieje i jest liczbą. \\
$b = +\infty $: "całka niewłaściwa I rodzaju"\\
$b \in R $: "całka niewłaściwa II rodzaju"\\
Analogiczna definicja jest określona dla całki niewłaściwej lewostronnej.
Do pewnego pojęcia można te pojęcia "mieszać", tj. rozważać sytuację, gdy funkcja jest określona na przedziale obustronnie domkniętym. [obrazek: 1]
To jeszcze nie jest takie złe - bo funkcja może być określona na dziurawym przedziale. Wtedy trzeba "pomieszać" całki niewłaściwe; chodzi o to, że wprowadza się jakiś punkt pośredni i zastanawiamy się nad sumą c.nw. od c do tego punktu, i z drugiej strony - od tego punktu do $c$. Można udowodnić (przy niewielkich założeniach), że suma tych dwóch całek nie zależy od wyboru punktu $c$. Czyli można rozważać skończoną liczbę takich punktów niewłaściwości, i to właśnie są mieszane całki niewłaściwe.
\section{Teoria Lebesgue'a}
Można udowodnić, że f. całkowalne w sensie Riemanna to takie funkcje ograniczone, które mają "nie za wiele" punktów nieciągłości - warunek konieczny i dostateczny na całkowalność w sensie Riemanna. Coś a'la nie musi być ciągła wszędzie, ale wszędzie bez ciągłości to lipa.
Całka Lebesgue'a jest równa całki w sensie Riemanna dla funkcji całkowalnych w sensie Riemanna. Zachowuje ona [Lebesgue'a] także własności liniowe całek Riemannowych. \\
Można całkować względem różnych miar. Przykładowo: miara Lebesgue'a - suma długości wszystkich (odcinków? części wykresu?) całek niewłaściwych.
\section{Zbieżność całek Lebesgue'a}
Teoria całek Lebesgue'a jest podobna do teorii szeregów, dlatego też interesuje nas pojęcie zbieżności \\
Przykłady \\
1. "pozorna niewłaściwość"
$f : [a;b) -> R$ ale można ją przedłużyć do \\
$f^{~} : [a;b] -> R $ tak, że $ f^{~} \in R$ \\
Fakt \\
Wówczas istnieje $\int_{a}^{b} f(x) dx$ i jest równa $\int[a,b] f^{~}(x)dx $ \\
2. $ \int_{1}^{+\infty} \frac{1}{x^\alpha} dx $ istnieje $ \forall \alpha > 0 $ i jest zbieżna wtw. $ \alpha > 1 $. Wówczas \[ \int_{1}^{+\inf} \frac{1}{x^\alpha} dx = \frac{1}{\alpha - 1} \]
Przypomina to zbieżność szeregu Riemanna - podobny warunek zachodził. Jednak wartość sumy szeregu i takiej całki jest oczywiście inna.
3. $ \int_{-\infty}^{0} e^x = 1 = -\int_{0}^{1} lnx dx $. Ta druga całka nie jest aż tak trywialna, ale Państwo sobie policzą w domu.
\section{Kryteria zbieżnosci całek Lebesgue'a}
Będą tylko dwa kryteria. Pierwsze jest uniwersalne, a drugie wyłącznie dla całek na przedziałach nieskończonych.
\subsection{Kryterium porównawcze całek Lebesgue'a}
Analogicznie dla kryterium porównawczego szeregów liczbowych; ale dodatkowe założenia. \\
Załóżmy, że $f_1, f_2 : [a;b) \to R $ - obie spełniają (*).
Jeśli dla każdego $ x \in [a,b) $
\[ 0 \leq f_1(x) \leq f_2(x) \]
oraz $ \int_a^b f_2 (t)dt $ jest zbieżna, to $\int_a^b f_1 (t)$ też jest zbieżna. \\
Dowód: \\
Dzięki założeniu istnieją obie całki niewłaściwe. Pewnie jest w skrypcie. No i tak jak w kryterium porównawczym. Można też sformułować podobne kryterium asymptotyczne - w jednym z zadań; więc raczej to nas nie obchodzi za bardzo.
\subsection{Kryterium Dirichleta}
W szeregach chcieliśmy zbieżność szeregów zadanych iloczynem. Teza brzmiała: $ \sum_{n=1}^{\infty} a_n b_n $ - zbieżny, o ile $ <a_n> $ monotoniczny i zbieżny do 0, oraz $\forall_n | \sum_{k=1}^n b_k | \leq M $, dla pewnego M. Natomiast w przypadku całek: \\
\[ f; g : [a; +\inf ) -> R \]
1. $f$ jest malejąca \\
2. $\lim_{x->+\inf} f(x) = 0 $ \\
3. $ \forall_{r \in [a; +\infty)} |\int_{[a;r]} g(x)dx| < M $ \\
Wówczas $ \int_n^{+\inf} f(x)g(x)dx$ jest zbieżna.
Dowodu brak - ale dowód się wyprowadza z pomocą przekształcenia Abela (w tw. z szeregów). Zamiast tego stosuje się jakieś twierdzenie o wartości średniej, którego nie znamy. 
\end{document}
